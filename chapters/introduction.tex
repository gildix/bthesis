Ehitusfüüsika on ehitusvaldkonna haru, mis käsitleb hoone toimivust füüsikaliste protsesside seisukohalt: soojus, niiskus, õhk, heli ja valgus,
seetõttu võib väita, et ehitufüüsikaga puutub oma elus kokku igaüks.
Ehitusfüüka valdkonna projekteerimise peamised eesmärgid on:
\begin{itemize}
    \item optimeerida hoone kütte ning jahutuskulud
    \item tagada hoones soojuslikku mugavust, niiskustingimusi ja sisekliima kvaliteeti tervikuna
    \item välistada mikrobioloogilist kasvu konstruktsioonides
    \item välistada veest ja niiskusest tekkivaid probleeme
    \item tagada hoonepiirete õhupidavust
    \item parandada akustilist kvaliteeti
\end{itemize}

Ehitusfüüsikavaldkond on oluline, sest see suures osas määratleb hoonete sisekliima kvaliteeti, teiste sõnadega tagab inimestele kvaliteetset 
elukeskkonda. Valesti projekteeritud hooned võivad muuhulgas avaldada negatiivset mõju inimeste tervisele või olla isegi ohtlikud. 
Seevastu õigesti projekteeritud hoone tagab kasutajale mugavusetunnet ja ka hoiab raha kokku minimeerides hoone kasutuskulusid.

Ressursside kallinemise olukorras sai ehitusfüüsikast eriti tähtis inseneriteaduse haru, sest muuhulgas see käsitleb hoone soojusliku 
toimivuse probleemi. See tähendab, et õigesti projekteeritud hoone talvel tarbib vähem energiat küttele ning suvel vastupidi -- jahutusele.

Ehitsfüüsikaga peab arvestama hoone elutsükli igal etapil - kavandamine, projekteerimine, ehitamine ja haldamine. Hoone kavandamisel 
määratakse planeeritavaid energiakulusid ja energiaklassi. Hoone projekteerimise faasis peavad ehitusfüüsikaga arvestama arhitektid, 
konstruktorid ja ka tehnosüsteemide projekteerijad, kes valivad õigete omadustega materjalid ning hindavad nende materjalide koosmõju 
konstruktsiooni toimimisele. Ehituse faasis peab ehitusfüüsikaga arvestama ehitusjuhid - kuigi ehitatakse tavaliselt projekti järgi, 
paraku peab ehituses ka operatiivselt võtta keerulisi otsuseid jooksvatest muudatustest keset ehitusprotsessi. Ja viimaseks peavad 
ehitusfüüsikat meeles hoidma ka hoone haldamisega tegelevad inimesed.

Probleemi teine külg on ehitusvaldkonna madal digitaliseerumise tase (ja konservatiivsus üldiselt). Viimastel aastatel on 
arendatud palju profesionaalseid tarkvarasid projekteerimise ja ehitusjuhtimise tarbeks, kuid ehitusfüüsika valdkonna 
tarkvara arendused on olnud väga tagasihoidlikud. Turul on olemas mõned üksikud tooted, kuid need on liiga keerulised ja võrdlemisi 
ebamugava kasutajaliidesega - sellise tarkvara sihtgrupp on teadusvaldkond. Ehitusinseneride töö hõlmab väga palju erinevaid asju 
ning on tavaliselt ajaliselt väga piiratud, mistõttu keerulise kasutajaliidesega ja tööpõhimõttega tarkvara kasutamine ei ole parim variant. 

Käesoleva töö eesmärk on välja töötada toodet, mis võimaldaks lahendada ehitusfüüsika valdkonna ülesandeid mugavalt 
ja operatiivselt. See võiks parandada olukorda, kus probleemide lahendamine jääb üldse erinevatel etapidel 
tegemata tarkvara või tarkvara kasutamise oskuste tõttu. See võiks olla ehitusinseneridele abivahendiks, mis ei vaja väga sügavat 
valdkonna tundmist, et teostada piisavas mahus arvutusi tagamaks ehitusprojekti või ehituse kvaliteeti ehitusfüüsika seisukohalt.
Ehitusfüüsika valdkond on lai ning lahendusi on tarvis leida väga paljudele probleemidele. Käesoleva töö raames keskendutakse esialgu vaid
ühe konkreetse probleemi lahendamisele, mis on ühtlasi ka kõige levinuim probleem - veeauru kondenseerumise riski hindamine
ehituskonstruktsioonides.