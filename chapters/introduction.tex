Ehitusfüüsika on ehitusvaldkonna haru, mis käsitleb hoone toimivust füüsikaliste protsesside seisukohalt: soojus, niiskus, õhk, heli ja valgus. Ehitusfüüsilise projekteerimise peamised eesmärgid on:
\begin{itemize}
    \item optimeerida hoone kütte ning jahutuskulud
    \item tagada hoones soojuslikku mugavust, niiskustingimusi ja sisekliima kvaliteeti tervikuna
    \item välistada mikrobioloogilist kasvu konstruktsioonides
    \item välistada veest ja niiskusest tekkivaid probleeme
    \item tagada hoonepiirete õhupidavust
    \item parandada akustilist kvaliteeti
\end{itemize}
Ehitusfüüsikavaldkond on oluline, sest see suures osas määratleb hoonete sisekliima kvaliteeti, teiste sõnadega tagab inimestele kvaliteetset elukeskkonda. Valesti projekteeritud hooned võivad
sealhulgas avaldada negatiivset mõju inimeste tervisele või olla isegi ohtlikud. Seevastu õigesti projekteeritud hoone tagab kasutajale mugavat sisekeskkonda ja ka hoiab raha kokku 
minimeerides hoone kasutuskulusid.

Ressursside kallinemise olukorras sai ehitusfüüsika eriti tähtsaks teaduse haruks, sest muuhulgas käsitleb hoone soojusliku toimivuse probleemi. See tähendab, et õigesti projekteeritud hoone
talvel tarbib vähem energiat küttele ning suvel vastupidi - jahutusele.

Ehitsfüüsikat peab arvestama hoone elutsükli igal etapil - kavandamine, projekteerimine, ehitamine ja haldamine. Hoone kavandamisel määratakse planeeritavaid energiakulusid ja energiaklassi.
Hoone projekteerimise faasis peavad ehitusfüüsikaga arvestama arhitektid, konstruktorid ja ka tehnosüsteemide projekteerijad valides õigete omadustega materjalid ning hindades nende
materjalide koosmõju konstruktsiooni toimimises. Ehituse faasis peab ehitusfüüsikaga arvestama ehitusjuhid - kuigi ehitatakse tavaliselt projekti järgi, paraku peab ehituses ka operatiivselt
võtta otsuseid jooksvatest muudatustest. Ja viimaseks peavad ehitusfüüsikaga arvestama ka hoone haldamisega tegelevad inimesed. 

Probleemi teine külg on ehitusvaldkonna väga madal digitaliseerumise tase (ja konservatiivsus tervikuna). Viimastel aastatel on arendatud väga palju profesionaalseid tarkvarasid
projekteerimise ja ehitusjuhtimise tarbeks, kuid ehitusfüüsika valdkonna tarkvara arendused on olnud väga tagasihoidlikud. Turul on olemas mõned üksikud tooted, kuid need on liiga 
keerulised ja väga ebamugava kasutajaliidesega - sellise tarkvara sihtgrupp on teadusvaldkond. Ehitusinseneride töö hõlmab väga palju erinevaid asju ning on tavaliselt ajaliselt väga
piiratud, mistõttu keerulise kasutajaliidesega ja tööpõhimõttega tarkvara kasutamine ei ole parim variant. 

Käesoleva töö eesmärk on välja töötada lahendust, mis võimaldaks teostada erinevaid ehitusfüüsikaga seotud analüüse mugavalt ja operatiivselt. See võiks parandada olukorda,
kus ehitusfüüsika probleemide lahendamine jääb üldse erinevatel etapidel tegemata tarkvara või tarkvara kasutamise oskuste tõttu.



\begin{enumerate}
    \item Mis on ehitusfüüsika
    \item Millised on ehitusfüüsika peamise ülesanded
    \item Miks on ehitusfüüsika oluline
    \item Kes peab ehitusfüüsikaga tegelema?
    \item Millised väljakutsed on viimasel ajal tekkinud ehitusfüüsika seisukohalt?
\end{enumerate}