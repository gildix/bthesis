Lõputöö analüütiline osa peab sisaldama visiooni kirjeldust ja skoobi
määramist põhjendusega. Metoodikat, kuidas kõige edukamalt lõpptulemuseni
jõuda. Missuguseid võimalusi on teada selle probleemi lahendamiseks. Määrata
funktsionaalsed ja mittefunktsionaalsed tingimused. Selle alusel kirjeldada
kasutajalood või siis teha tegevuse plokkskeem. Leida töövahendid
põhjendatult: milliste kriteeriumite alusel tuleb valik teha, milline lahendus
valitakse selle töö puhul, kindlasti põhjendada.

Analüüs peab sisaldama vastuseid küsimustele:
\begin{enumerate}
    \item visiooni kirjeldust ja skoobi määramist põhjendusega
    \item missuguseid lahendusi te teate oma probleemi lahendamiseks
    \item milliste kriteeriumite alusel tuleb valik teha
    \item millise lahenduse valisite teie ja põhjendage seda
\end{enumerate}

\section{Nõuete defineerimine}
Nõute määramine rakendusele.

\section{Tehnoloogiate ja meetodite valik}
Osa, kus käsitletakse tehnoloogiaid ja arendusmetoodikate valikut.

\section{Veebirakenduse arhitektuur}
Osa, kus käsitletakse kavandatava rakenduse arhitektuuri planeerimist.

\section{Andmebaasi projekteerimine}
Osa, kus käsitletakse andmebaasi projekteerimist.

\section{Kasutajaliidese disain}
Osa, kus käsitletakse kasutajaliidest ja selle kavandamist