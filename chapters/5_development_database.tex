\section{Andmebaas}
Vastavalt käesoleva töö osale \ref{analysis_database_subsection} projekti realiseerimiseks
on valitud PostgreSQL versioon 16. Andmebaasi juhtprogramm paigaldatakse samale serverile, milles 
käivitatakse ka ülejäänud infosüsteemi osad. PostgreSQL on Ubuntus paigaldamiseks saadaval
\textit{apt} repositooriumil. Peale paigaldamist redigeeritakse PostgreSQL konfiguratsiooni - 
\textit{postgresql.conf} ja \textit{pg hba.conf}, seadistades porti, mida kuulatakse (5432) ning 
IP aadressid, millistelt võetakse päringuid vastu.

Turvalisuse tagamiseks luuakse arendatavale rakendusele eraldi PostgreSQL kasutajat parooliga, 
piirates kasutaja õigusi ühe konkreetse andmebaasiga, mis on rakenduse funktsioneerimiseks tarvis.
Andmebaasile tehakse iganädalased varukoopiad, mida salvestatakse serveril. Selle jaoks luuakse skripti ning 
määratakse Ubuntu Cron regulaarse tööna.



