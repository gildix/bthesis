\label{chapters:feedback}
Hinnang arendatud infosüsteemile oli küsitud sihtgrupi esindajatelt. Sihtgrupi erinevatest valdkondadest (arhitektuurne projekteerimine,
konstruktsiivne projekteerimine, ehitusfirma kvaliteedijuhtimine, ehitusprotsessi juhtimine, omaniku järelevalve teostaja) olid leitud
esindajad, kes oli valmis infosüsteemi läbiproovida ja tagasisidet anda.

Hinnangu moodustamiseks oli korraldatud minimaalse väärtusliku funktsionaalsuse proovimine ja tagasiside korjamine. Proovijatele oli saadetud link
veebirakendusele, registreerimise võti ja lihtne stsenaarium, mida võiks katsuda rakenduses läbi mängida:
\begin{itemize}
    \item registreerida kasutajaks kasutades saadetud registreerimise võtit
    \item logida sisse
    \item luua ehitusmaterjal ja lisada sellele arvutuseks vajalikud omadused (vihje: Puit, tihedus: 800g/m3, 
    soojuserijuhtivus: 0.3 W/mK, difusiooni takistustegur 20)
    \item avada soojusjuhtivuse kalkulaator ja modelleerida konstruktsiooni, mis sisaldab sisestatud materjali
    \item proovida modelleerida konstruktsiooni, mis koosneb kolmest kihist (betoon 100 mm, mineraalvill 150 mm, betoon 80 mm),
    uurida kas esineb kondensaadi tekkimise tõenäosus ja millesel aastaajal
\end{itemize} 

Toodud stsenaariumi alusel oli pakutud anda hinnangut järgmistele aspektidele:
\begin{enumerate}
    \item kasutajaliidese üldine kasutusmugavus
    \begin{enumerate}
        \item stsenaariumi läbimine läks sujuvalt
        \item vajalike kasutajaliidese elementide leidmisega (nupud, väljad, lehed) ei olnud raskusi
        \item veebirakenduse käitumine oli ootuspärane
    \end{enumerate}
    \item veebirakenduse äriloogika 
    \begin{enumerate}
        \item ehitusmaterjali lisamise vorm võimaldab salvestada kõiki vajalikke (käesoleva süsteemi kontekstis) andmeid materjalist
        \item ehitustmaterjalide nimekiri on ülevaatlik, sellest on näha kõiki salvestatud andmeid
        \item kalkulaatori lehel arvutuse teostamiseks tegevuste järjekord on intuitiivselt aru saadav
        \item arvutuse tulemuste esitamine kalkulaatori lehel on piisavalt selge ja arusaadav
    \end{enumerate}
    \item üldine kontseptsioon
    \begin{enumerate}
        \item arendatud infosüsteem leiab teie töös kasutust
        \item näete rakenduse ideel edasise arengu potentsiaali
    \end{enumerate}
\end{enumerate}

Kõik aspektid hinnati Likerti neljaarvulise skaalaga, valides järgmistest hinnetest: "ei ole nõus", "pigem ei ole nõus", "pigem nõus" ja "nõus". 
Samuti küisiti ettepanekuid veebirakenduse edasise arengu vajadusest ja suunast. 


