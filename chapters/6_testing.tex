\label{chapters:testing}
Infosüsteemi serveriosa kõige kriitilsema funktsionaalsuse testitakse automaattestimisega.
Kõigepealt peab automaatteste luua materjalide lisamise testimiseks. Testimisega kontrollitakse
järgmised materjalide (ja nende omaduste) lisamisega ja turvalisusega seotud väited:
\begin{itemize}
    \item kasutaja A näeb avalikke materjale
    \item kasutaja A saab luua uus materjal
    \item kasutaja B ei näi kasutaja A poolt loodud materjale
    \item sisse logimata kasutaja näev avalikud materjalid 
    \item sisse logimata kasutaja ei näe kasutaja A ja kasutaja B poolt lisatud materjalid
    \item kasutaja saab lisada materjalile omadust
    \item kasutaja saab redigeerida materjali omadust
    \item kasutaja näeb omadusi, mis on materjali puhul olemas
\end{itemize}

Samamoodi testitakse ka arvutusteenust. Arvutusteenuse testimisel peab kontrollima järgmiseid väiteid:
\begin{itemize}
    \item arvutuse lähteandmetes näeb kasutaja tema poolt sisestatud materjalid 
    \item arvutuse lähteandmetes näeb kasutaja vaid need materjalid, millistele on lisatud
    arvutuse teostamiseks nõutud omadused
    \item teenus tagastab õiged arvutuse tulemused
\end{itemize}

Samuti kontrollitakse, et ebaõnnestunud päringutele tagastab alati teenus adekvaatset ja informatiivset 
vastust (nt ligipääsu puudumine, vigased päringuga saadetud andmed).

Kasutajaliidese testimist viiakse läbi manuaalse testimisena. Testimise aluseks on peatükis \ref{chapter:analysis} 
toodud kasutajalugud.