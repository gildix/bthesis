\section{Serveriosa}

\textbf{TODO:Märksõnad}


Kihiline arhitektuur
\begin{itemize}
    \item andmebaasi ja domeeni kiht
    \item andmete ligipääsu kiht
    \item äriloogika kiht
    \item andmete representeerimise kiht
\end{itemize}

\textbf{Domain} (andmebaasi) kihis realiseeritakse projekteeritud andmebaasi mudel vastavate klassidena (\textit{Domain.App.Material},
\textit{Domain.App.MaterialCategory}, \textit{Domain.App.MaterialProperty} jne). Igas klassis määratakse vajalikud omadused ja
olemite vahelised seosed. Domain kiht moodustab andmebaasi konteksti (DbContext), mis on (\textit{Entity Framework Core} raamistiku klass,
mille kaudu raamistik suhtleb andmebaasiga.


\textbf{DAL} andmete ligipääsu kihis teostatakse tööd andmebaasi konstektsti objektiga (DbContext):  andmeid küsitakse Entity Framework Core-ist ja 
vastavalt ka salvestatakse \textit{DbContext}-isse. Kiht on jagatud üksusteks -- reposiooriumideks (\textit{Repository}), mis on 
baasimplementatsioonis kujutavad ennast klassikalist CRUD-tüüpi repositooriumi. Samuti repositooriumis 
teostatakse andmetele ligipääsu kontrolli kasutaja ID alusel -
nt meetod \textit{GetAllAsync(Guid uid, bool includePublic = false, String Include what, String)} küsib andmebaasist kõik olemid, 
mis kuuluvad kasutajale ID-ga \textit{uid} ning valikuliselt lisatakse ka olemeid, mis on ette nähtud ühiskasutuseks. 
Repositooriumiteks jagamist teostatakse andmemudeli ülesehituse alusel -- iga olemi jaoks on eraldi repositoorium. Selleks, et
äriloogika kihis erinevates teenustes oleks repositooriumite kasutamine paindlik, moodustatakse kõikidest repositooriumitest üks
objekt (\textit{UOW - Unit Of Work}), mille kaudu võib juurde pääseda igale repositooriumile.

Kui rakendus vajab mõne olemi puhul keerulisemat repositooriumi loogikat, siis baasfunktsionaalsus saab olla üle kirjutatud
või laiendatud. Näiteks, materjalide (\textit{Domain.App.Material}) puhul on otstarbekam kohe agregeerida andmeid, 
mis puudutavad materjalide omadusi ((\textit{Domain.App.MaterialProperty}) ja (\textit{Domain.App.Property})), siis
äriloogika kihis ei pea tegema palju lisapäringudeta, et teostada vajalikke arvutusi ja tegevusi.


\textbf{BLL} äriloogika kihis toimub kogu põhiline töö, mis on seotud vahetult rakenduse funktsionaalsust puudutava loogikaga.
Äriloogika kihi üksuseks on teenus (\textit{Service}). Teenuste moodustamise loogika suuresti vastab \textbf{DAL} kihi jagamise loogikale,
et võimalikult isoleerida erinevad teenused omavahelt. On ka teenused, mille eesmärk on erinevatest repositooriumitest andmete agregeerimine --
näiteks, arvutuste teenus (\textit{CalculationController}), mille otstarve on arvutusteks vajalike andmete komplekteerimine kasutajaliidesele
saatmiseks, kasutajaliideselt tulnud arvutuse päringu töötlemine, arvutuste teostamine ja arvutuse tulemuste tagastamine. 





