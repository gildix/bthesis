\section{Andmebaasi projekteerimine}
Arendatavas infosüsteemis kasutatakse relatsioonilist andmebaasi, mille puhul andmeb on koondatud
tabelitesse. 
Andmebaasi primaarvõtmeteks kasutatakse GUID võtmed, mis on 128-biti pikkusega sõne.
GUID on piisavalt pikk, et võtme unikaalsus oleks piisava tõenäosusega tagatud. Samuti GUID on genereeritud
raamistikuga, mis on kiirem \cite{guid_definition}. Kuna infosüsteemis on kasutusel ORM Entity Framework, 
siis andmebaasi skeemi luuakse automaatselt koodis ehitatud mudeli järgi. Vaatamata sellele, enne 
andmemudeli implementeerimist koodis peab läbi mõtlema selle ülesehitust ja loogilisi seoseid andmemudeli
objektide vahel.
Selle jaoks on koostatud andmebaasi ERD diagrammil, millel esitatakse andmemudelis olevad seosed graafilisel
kujul. Andmemudeli diagramm on esitatud pildil X.