\label{chapters:introduction}
Ehitusfüüsika on ehitusvaldkonna haru, mis käsitleb hoonet füüsikaliste nähtuste seisukohalt: soojus, niiskus, õhk, heli ja valgus
Võib väita, et ehitufüüsikaga puutub oma elus kokku igaüks, kuna hoone sisekliima mugavus, küttarved ja müra, mis kostub tänavalt tuppa
on samuti lahutamatult seotud ehitusfüüsikaga.

Ehitusfüüka valdkonna projekteerimise peamised eesmärgid on:
\begin{itemize}
    \item optimeerida hoone kütte ning jahutuskulud
    \item tagada hoones soojuslikku mugavust, niiskustingimusi ja sisekliima kvaliteeti tervikuna
    \item välistada mikrobioloogilist kasvu konstruktsioonides
    \item välistada veest ja niiskusest tekkivaid probleeme
    \item tagada hoonepiirete õhupidavust
    \item parandada akustilist kvaliteeti
\end{itemize}

Ehitusfüüsikavaldkond on oluline, sest see suures osas määratleb hoonete sisekliima kvaliteeti, teiste sõnadega tagab inimestele kvaliteetset 
elukeskkonda. Valesti projekteeritud hooned võivad avaldada negatiivset mõju inimeste tervisele ning seevastu õigesti projekteeritud 
hoone tagab kasutajale mugavusetunnet ja ka hoiab raha kokku minimeerides hoone kasutuskulusid. Ressursside kallinemise olukorras sai 
ehitusfüüsikast eriti tähtis inseneriteaduse haru, sest muuhulgas see käsitleb hoone soojusliku 
toimivuse probleemi. See tähendab, et õigesti projekteeritud hoone talvel tarbib vähem energiat küttele ning suvel -- jahutusele.

Ehitsfüüsikaga peab arvestama hoone elutsükli igal etapil -- kavandamine, projekteerimine, ehitamine ja haldamine. Hoone kavandamisel 
arvutatakse välja planeeritavad energiakulud ja määratakse hoone energiaklassi. Hoone projekteerimise faasis peavad ehitusfüüsikaga arvestama arhitektid, 
konstruktorid ja ka tehnosüsteemide projekteerijad, kes valivad õigete omadustega materjalid ning hindavad nende materjalide koosmõju 
konstruktsiooni toimivusele. Ehituse faasis peab ehitusfüüsikaga arvestama ehitusjuhid: kuigi ehitatakse tavaliselt projekti järgi, 
paraku peab ehituses ka operatiivselt võtta keerulisi otsuseid jooksvatest muudatustest keset ehitusprotsessi. Ja viimaseks peavad 
ehitusfüüsikat meeles hoidma ka hoone haldamisega tegelevad inimesed.

Probleemi teine külg on ehitusvaldkonna madal digitaliseerumise tase (ja konservatiivsus üldiselt). Viimastel aastatel on 
arendatud palju profesionaalseid tarkvarasid projekteerimise ja ehitusjuhtimise tarbeks, kuid ehitusfüüsika valdkonna 
tarkvara arendused on olnud väga tagasihoidlikud. Turul on olemas mõned üksikud tooted, kuid need on liiga keerulised ja võrdlemisi 
ebamugava kasutajaliidesega -- sellise tarkvara sihtgrupp on teadusvaldkond. Ehitusinseneride töö hõlmab väga palju erinevaid asju 
ning on tavaliselt ajaliselt väga piiratud, mistõttu keerulise kasutajaliidesega ja tööpõhimõttega tarkvara kasutamine ei ole parim variant. 

Käesoleva töö eesmärk on välja töötada platvormi, mis sisaldaks erinevad tööriistad, millega oleks võimaliks lahendada ehitusfüüsika valdkonna erinevaid ülesandeid mugavalt 
ja operatiivselt sellisel tasemel, mis oleks ehitusinseneridele piisav. See võiks parandada olukorda, kus ehitusfüüsika probleemidega tegelemine
jääb üldse erinevatel etapidel tegemata tarkvara puudumise või tarkvara kasutamiseks ebapiisavate oskuste tõttu. 
See oleks ehitusinseneridele abivahendiks, mis ei vaja väga sügavat valdkonna tundmist, et teostada piisavas mahus arvutusi 
tagamaks ehitusprojekti või ehituse kvaliteeti ehitusfüüsika seisukohalt. 

Ehitusfüüsika valdkond on lai ning lahendusi on tarvis leida väga paljudele probleemidele -- detailsem soojusjuhtivuse arvutus erinevatele hoone sõlmedele,
energiamärgise või ventilatsiooni- ja õhuvahetusega seotud arvutused, ja ka palju muud. Käesoleva töö raames on mahu piiramise mõttes mõistlik
keskenduda ühe konkreetse probleemi lahendamisele ja töötada välja selleks tööriist. Lahendatavaks probleemiks võiks olla konstruktsiooni
niiskustehnilise toimivuse kalkulaator, millega saaks lihtsamal viisil hinnata kondensaadi tekkimise riski konstruktsiooni kihtides.
Tegemist on ehitusinseneridele vajaliku tööriistaga, mille arendamine võiks samuti olla infotehnoloogia valdkonna seisukohalt
huvitavaks väljakutseks.

Püstitatud probleemi lahenduse lähteülesanne suures osas toetub autori teadmistele ja kogemustele ehituse valdkonnas. Töö autor omab magistrikraadi
tööstus- ja tsiviilehituse erialal ning on rohkem kui 10 aastat töötanud ehitusvaldkonnas erinevates ametites, seetõttu aimab sihtgrupi vajadustest. 