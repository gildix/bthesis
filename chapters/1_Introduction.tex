\label{chapters:introduction}
Ehitusfüüsika on ehitusvaldkonna haru, mis käsitleb hoonet füüsikaliste nähtuste seisukohalt: soojus, niiskus, õhk, heli ja valgus
Ehitufüüsikaga puutub oma elus kokku igaüks, kuna hoone sisekliima mugavus, küttarved ja müra, mis kostub tänavalt tuppa
on samuti lahutamatult seotud ehitusfüüsikaga.

Ehitusfüüsika valdkonna projekteerimise peamised eesmärgid on:
\begin{itemize}
    \item optimeerida hoone kütte ning jahutuskulud
    \item tagada hoones soojuslikku mugavust, niiskustingimusi ja sisekliima kvaliteeti tervikuna
    \item välistada mikrobioloogilist kasvu konstruktsioonides
    \item välistada veest ja niiskusest tekkivaid probleeme
    \item tagada hoonepiirete õhupidavust
    \item parandada akustilist kvaliteeti
\end{itemize}

Ehitusfüüsikavaldkond on oluline, sest see suures osas määratleb hoonete sisekliima kvaliteeti, teiste sõnadega tagab inimestele kvaliteetset 
elukeskkonda. Valesti projekteeritud hooned võivad avaldada negatiivset mõju inimeste tervisele \cite{rokka_hygrothermal} ning seevastu õigesti projekteeritud 
hoone tagab kasutajale mugavusetunnet ja ka hoiab raha kokku minimeerides hoone kasutuskulusid \cite{kalamees_phd}. Ressursside kallinemise olukorras sai 
ehitusfüüsikast eriti tähtis inseneriteaduse haru, sest muuhulgas see käsitleb hoone soojusliku toimivuse probleemi. See tähendab, et 
õigesti projekteeritud hoone talvel tarbib vähem energiat küttele ning suvel -- jahutusele.

Ehitusfüüsikaga peab arvestama hoone elutsükli igal etapil -- kavandamine, projekteerimine, ehitamine ja haldamine. Hoone kavandamisel 
arvutatakse välja planeeritavad energiakulud ja määratakse hoone energiaklassi \cite{energia_miinimum}. Hoone projekteerimise faasis peavad ehitusfüüsikaga arvestama arhitektid, 
konstruktorid ja ka tehnosüsteemide projekteerijad, kes valivad õigete omadustega materjalid ning hindavad nende materjalide koosmõju 
konstruktsiooni toimivusele. Ehituse faasis peab ehitusfüüsikaga arvestama ehitusjuhid: kuigi ehitatakse tavaliselt projekti järgi, 
paraku peab ehituses ka operatiivselt võtta keerulisi otsuseid jooksvatest muudatustest keset ehitusprotsessi. Ja viimaseks peavad 
ehitusfüüsikat meeles hoidma ka hoone haldamisega tegelevad inimesed.

\section{Probleemi olemus}
\label{chapters:problem_statement}
Ehitusvaldkonna oluline puudus on madal digitaliseerimise tase \cite{digitalization_index} ja konservatiivsus üldiselt. See puudutab ka ehitusfüüsika valdkonda. 
Viimasel aastakümnel on  arendatud palju professionaalseid tarkvarasid projekteerimise ja ehitusjuhtimise tarbeks, kuid  käesoleva töö 
peatükis \ref{chapter:problem_statement} läbiviidud analüüsist järeldub, et ehitusfüüsika valdkonnas digitaalsete lahenduste arendamine oli tagasihoidlik. 
Euroopa turul on olemas mõned üksikud lahendused (toodud peatükis \ref{chapter:problem_statement}), mis on üsna keerulised -- sellise tarkvara sihtgrupp on teadusvaldkond.
Ehitusinseneride töö on tavaliselt ajaliselt piiratud, mistõttu keerulise kasutajaliidesega ja tööpõhimõttega tarkvara kasutamine ei ole alati sobilik variant. 
Lisaks, puudub tarkvara, mis oleks kohalikule turule adapteeritud (eestikeelse kasutajaliidese ja kliiamaandmete olemasolu).

Käesoleva töö eesmärk on välja töötada platvormi, mis sisaldaks erinevaid tööriistu, millega oleks võimalik operatiivselt lahendada ehitusfüüsika valdkonna ülesandeid
tuginedes aktuaalsetele andmetele ning tagades tulemuste mahu ja kvaliteeti ehitusinseneridele piisaval tasemel. 
Sellega võiks mõjutada positiivses suunas olukorda, kus ehitusfüüsika probleemidega tegelemine jääb erinevatel etappidel tegemata tarkvara puudumise või tarkvara kasutamiseks ebapiisavate oskuste tõttu. 
Kavandatav platvorm oleks ehitusinseneridele abivahendiks, mis ei vaja sügavat valdkonna tundmist, et teostada piisavas mahus arvutusi 
tagamaks ehitusprojekti või ehitise kvaliteeti ehitusfüüsika seisukohalt. 

Ehitusfüüsika valdkond on lai ning lahendusi on tarvis leida paljudele probleemidele -- detailsem soojusjuhtivuse arvutus erinevatele hoone sõlmedele,
energiamärgise või ventilatsiooni- ja õhuvahetusega seotud arvutused, ja ka palju muud. Käesoleva töö raames on mahu piiramise mõttes mõistlik
keskenduda ühe konkreetse probleemi lahendamisele ja töötada välja selleks tööriist. Lahendatavaks probleemiks on valitud konstruktsiooni
niiskustehnilise toimivuse kalkulaator, millega saaks lihtsamal viisil hinnata kondensaadi tekkimise riski piirdekonstruktsiooni kihtides.
Tegemist on ehitusinseneridele vajaliku tööriistaga, mille arendamine võiks samuti olla infotehnoloogia valdkonna seisukohalt
huvitavaks väljakutseks.

Püstitatud probleemi lahenduse lähteülesanne suures osas toetub autori teadmistele ja kogemustele ehituse valdkonnas. Töö autor omab magistrikraadi
tööstus- ja tsiviilehituse erialal ning on rohkem kui 10 aastat töötanud ehitusvaldkonnas erinevates ametites, seetõttu aimab sihtgrupi vajadustest.

\section{Töö sisu}
Töö koosneb kaheksast peatükist. Käesolevas sissejuhatuse peatükis \ref{chapters:introduction} tuuakse üldine teave valitud valdkonnast, valiku 
põhjendused ning konkretiseeritakse ülesande püstitust. Metoodika peatükis \ref{chapters:metodology} kirjeldatakse püstitatud probleemile lahenduse leidmiseks
vajalikud sammud. Peatükis \ref{chapters:problem_statement} uuritakse sissejuhatuse peatükis \ref{chapters:introduction} püstitatud probleemi detailsemalt, 
selgitades välja nõutud lahenduse sisu ja eksisteerivaid lahendusi. Kavandatava lahenduse analüüsi peatükis \ref{chapters:analysis} tuuakse välja nõuded 
infosüsteemile, valitakse tehnoloogiaid erinevate infosüsteemi osade realiseerimiseks ning koostatakse kasutajaliidese disaini projekt. Infosüsteemi arendamise
peatükis \ref{chapter:development} kirjeldatakse infosüsteemi osade realiseerimise põhilised aspektid ning testimise peatükist \ref{chapters:testing} räägitakse
infosüsteemi testimise põhimõtetest. Osas \ref{chapters:feedback} kavandatakse ja viiakse läbi infosüsteemi proovikasutamist sihtgrupi esindajate poolt ning 
analüüsitakse tulemusi. Kokkuvõtte peatükis \ref{chapters:summary} antakse hinnangut läbiviidud tööle ning hinnangule tuginedes tehakse ettepanekuid infosüsteemi
edasise arengu suunast ja strateegiast.