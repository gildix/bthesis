\label{chapters:metodology}
Probleemi lahendamist alustatakse olemasolevate lahenduste otsimisest ja analüüsimisest. Iga lahenduse puhul
tuuakse välja tugevad küljed ja puudused, arvestades planeeritava toote kontseptsioonist ja sihtgrupist. Samuti
tehakse erinevate lahenduste hinnavõrdlust. 
Lähtuvalt lahenduste analüüsi tulemustest defineeritakse konkreetsed nõuded kavandatavale 
infosüsteemile, millest lähtutakse infosüsteemi tehniliste lahenduste projekteerimisel.
Antud kohas määratakse ka toote MVP, mis oleks lõputöö mahu kohane.

Kui nõuded infosüsteemile on paika pandud, valitakse infosüsteemi ehitamise tehno-loogiaid -- kasutajaliides,
serveriosa ja andmebaas. Tehnoloogiate all mõeldakse konkreetsed programmeerimiskeeled ja raamistikud. Samuti
lähtuvalt infosüsteemi nõuetest projekteeritakse kasutajaliidese disainilahendust.

Seejärel kavandatakse nii üldist infosüsteemi arhitektuuri (kuidas infosüsteemi osad omavahel töötavad, millised andmed
kasutajaliidese ja serveriosa vahel liiguvad), kui ka arhitektuursed lahendust iga infosüsteemi osale eraldi 
(näiteks: serveriosa ja kasutajaliidese struktuur).

Seejärel planeeritakse arenduse protess. Kavandatav funktsionaalsus jagatakse kasutaja-lugudeks, mida gruppeeritakse
\colorbox{BurntOrange}{featuurideks ja \textit{epic}-uteks}. Kasutajalugudest moodustatakse tehnilised ülesanded,
mida võetakse aluseks koodu kirjutamisel. 

{\textbf{TODO: kuidas selliseid asju õigesti kirjutada?}

Kui MVP funktsionaalsus on saavutatud, siis toodet antakse mitmele sihtgrupi esindajatele testimiseks 
ja tagasiside saamiseks. Tagasiside alusel kavandatakse edasist arendusprotsessi.


