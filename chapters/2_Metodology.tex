\label{chapters:metodology}
Probleemi lahendamist alustatakse olemasolevate lahenduste otsimisest ja analüüsimisest. Iga lahenduse puhul
tuuakse välja tugevad küljed ja puudused, arvestades planeeritava toote kontseptsioonist ja sihtgrupist. Samuti
tehakse erinevate lahenduste hinnavõrdlust, kui hinnakiri on avalikult kättesaadav.
Lähtuvalt lahenduste analüüsi tulemustest defineeritakse konkreetsed nõuded kavandatavale 
infosüsteemile, millest lähtutakse infosüsteemi tehniliste lahenduste projekteerimisel.
Lähtuvalt nõuetest valitakse infosüsteemi ehitamise tehnoloogiaid -- kasutajaliides,
serveriosa ja andmebaas. Tehnoloogiate all mõeldakse konkreetsed programmeerimiskeeled ja raamistikud.
Iga infosüsteemiosa puhul tuuakse võrdlust ja põhjendatakse valikult. Samuti lähtuvalt infosüsteemi 
nõuetest projekteeritakse kasutajaliidese disainilahendust.

Seejärel kavandatakse nii üldist infosüsteemi arhitektuuri (kuidas infosüsteemi osad omavahel töötavad, millised andmed
kasutajaliidese ja serveriosa vahel liiguvad), kui ka detailsemad lahendused iga infosüsteemi osale eraldi 
(näiteks: serveriosa ja kasutajaliidese struktuur). Arenduse protsessi planeerimisel kavandatav funktsionaalsus jagatakse kasutaja-lugudeks, mida grupeeritakse
featuurideks (\textit{feature}) ja eeposteks (\textit{epic}). Kasutajalugudest moodustatakse tehnilised ülesanded,
mida võetakse aluseks koodi kirjutamisel. 

Töömahu piiramiseks valitakse planeeritud funktsionaalsusest minimaalse elujõulise 
toote funktsionaalsust (MVP) \cite{lean_startup_book}, mida viiakse ellu käesoleva lõputöö raames. Kui MVP funktsionaalsus 
on realiseeritud, antakse toodet erinevates ametites töötavatele sihtgrupi esindajatele toote
proovimiseks ja tagasiside saamiseks. Tagasisidele tuginedes antakse hinnangut tehtud tööle ja tehakse ettepanekuid
toote edasiseks arenguks.


