\section{Kasutajaliides}
Vastavalt peatükile \ref{analysis_interface_subsection} kasutajaliidese realiseerimise tehnoloogiaks
on valitud React.Js raamistik.

Kasutajaliidese rakendus koosneb \textit{React JSX.Element} komponentidest ja teenustest.
Komponendid tagavad rakendusele nõutud funktsionaalsust ja välimust ning teenused vahendavad andmeid komponentide 
ja serveriosa vahel.

Materjali loomiseks ja redigeerimiseks on luuakse vorm, mis täidab samaaegselt mõlemat rolli. 
Kui vormi avamisel oli edasi antud materjalid ID, siis laetakse vastav materjal andmebaasist redigeerimiseks, 
kui ID ei olnud - kohe vormi avamisel luuakse päringuga uus materjal mustandi staatusega.
Materjali vorm koosneb kahest React komponendist - vormi põhi ja materjali omaduse plokk.
Vormi põhi on suurem komponent \textit{MaterialForm}, mille kaudu redigeeritakse materjali staatilised 
väljad:

\begin{itemize}
    \item nimetus (\textit{title}) -- tekstiväli,
    \item materjali tüüp (\textit{materialCategoryId}) -- valiklist materjalide kategooriatest, 
    \item tootja (\textit{manufacturerId}) -- valiklist materjalide tootjatest,
    \item allikas (\textit{source}) -- tekstiväli,
    \item viide allikale (\textit{link}) -- tekstiväli,
    \item kommentaar (\textit{comment}) -- tekstiväli.
\end{itemize}

Materjalide omaduste (nt tihedus, soojusjuhtivus) arv on muutuv -- mõne materjali puhul võivad olla salvestatud
kõik omadused, teise materjali puhul vaid mõned neist. Lisaks sellele, tulevikus infosüsteemisse võib olla lisatud
võimalus salvestada muud omadused, mis on tarvis teiste arvutuste jaoks. Sellest tingituna peab olema võimalus kuvada
ja redigeerida need omadused, mis on materjali puhul määratud. Lisaks peavad olema näidatud ka omadused,
mida on võimalik lisaks määrata. Materjali omadusi redigeerimiseks tehakse väiksem taaskasutatav 
komponent \textit{MaterialPropertyCard}. Komponendil on järgmised väljad:
\begin{itemize}
    \item väärtus (\textit{value}) -- materjali omaduse arvuline väärtus,
    \item tõendatud (\textit{verified}) -- boolean väärtus,
    \item allikas (\textit{source}) -- tekstiväli.
\end{itemize}

Komponendile \textit{MaterialPropertyCard} antakse materiali omadusi sisaldavast objektist
\textit{materjal.materialProperties} edasi üks objekt ja funktsioon, mis töötleb sisendi andmete muutmist
ja tagastab põhikomponendisse sisestatud väärtused. Põhikomponendis uuendatakse olekut ja vastavalt
saadetakse ka päringuid materjali uuendamiseks andmebaasis.
\textit{MaterialPropertyCard} välimuse seisukohalt on kaks olekut sõltuvalt sellest, kas vastav omadus on materjalile lisatud
või vastav omadus ei ole veel materjalile määratud ning seda on võimalik määrata -- pilt X.
