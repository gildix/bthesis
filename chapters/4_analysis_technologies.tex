\section{Tehnoloogiate valik}
Tehnoloogiate valimisel lähtutakse kaasaaegsetest veebirakendamise ehitamise pritsiibidest, arvestatakse
rakenduste loogika keerukust, võimalike kasutajate hulka, säilitavate andmete mahtusid ja infosüsteemi
edasise arengu perspektiive.

Veebirakendusel peab olema selgelt eristatud serveriosa ja kasutajaliides. Vajadusel saab tulevikus 
implementeerida ka teised kasutajaliidesed, mis töötavad sama serveriosaga (näiteks: mobiilrakendus).
Selleks, et süsteem saaks võimalikult pikemat aega töötama ilma tehnoloogiate uuenduste vajaduseta,
peab võtma võimalikult uued, aga samas ka stabiilsed lahendused. 

Andmevahetus kasutajaliidese ja serveriosa vahel toimub JSON (JavaScript Object Notation) formaadis.
JSON on JavaScript-i põhine andmevahetuse formaat, mis representeerib JavaScript-i andmeobjektid
tekstilisel kujul. 

\subsection{Kasutajaliides}
Kasutajaliides implementeeritakse üheleherakendusena (SPA). SPA tehnoloogia võimaldab minimeerida andmevahetuse mahtu:
serverilt küsitakse ja vastavalt kliendile saadetakse ainult need andmed, mida on hetkel tarvis. Arendatava infosüsteemi
kontekstis see on oluline, sest kõiki arvutusi tehakse serveril ning kliendile saadetakse andmed, mis on 
vajalikud tulemuste näitamiseks. Iga uus tegevus kasutajaliideses, mis mõjutab tulemusi (uue kihi lisamine, 
kihtide järjekorra muutmine, arvutuse parameetrite muutmine jms.), tähendab uut päringut serverile. 
Samuti SPA tehnoloogia võimaldab muuta lehe sisu dünaamilisel viisil -- uuendatakse vaid lehe teatud 
osa ilma kogu lehekülje ümberlaadimise vajaduseta. See on ka oluline, kuna tulemusi peab uuendama 
kohe peale arvutuse lähteandmete muutmist. Üheleherakenduste implementeerimiseks kasutatakse JavaScript
programmeerimiskeeli rakenduse dünaamilise loogika juhtimiseks ning HTML ja CSS lehtede kujundamiseks.

Kuigi on võimalik implementeerida loogikat kasutades puhtat JavaScript koodi, tänapäeval seda tehakse
harva. On olemas erinevad raamistikud, mis oluliselt lihtsustavad rakenduse ehitamise protsessi, kuid 
nõuavad ka spetsiifilisi teadmisi. Raamistiku kasutuselevõtt olulisel määral vähendab koodi kirjutamist,
kuna raamistik ise haldab palju asju, mis on seotud \textit{Routing}-uga, turvalisusega, komponentide
genereerimise ja uuendamisega. Spetsiifiliste asjade jaoks kasutatakse eraldi pluginaid ja teeke. Näiteks
päringute saatmise ja serveri vastuse töötluseks kasutatakse \textit{Axios} -- teek, mida saab kasutada 
erinevate raamistikutega.

Üheleherakenduse implementeerimiseks kõige sobilikud JavaScript raamistikud on \textit{React}, \textit{Vue.js}
ja \textit{Angular}. Kõikidel raamistikutel on oma eripärad alates projekti arhitektuurist kuni 
koodi süntaksini. 

\textit{React} on laialt levinud \textit{front-end} teek, mis kasutab JavaScript programmeerimiskeelt. 
Lehe šablooni kujundamiseks kasutatakse JSX (JavaScript XML). JSX on JavaScript-i laiendus, mis võimaldab
sisestada HTML koodi JavaScript-i programmi. Rakendus koosneb React-elementidest, mille uuendamisega teek
tegeleb ise. Elemendid on taaskasutatavad ning nendele antakse andmeid edasi andmeobjektide kujul (\textit{props}).
Kuna lahendus on populaarne -- selle kasutamise kohta on kogutud palju teavet ja kogemust Internetis, mistõttu
probleemide tuvastamine ja lahenduste leidmine on piisavalt lihtne. Lisaks sellele eksisteerib palju pluginaid
ja teeke, mida saab React raamistikuga ühendada funktsionaalsuse laiendamiseks. 

\textit{Vue.js} on MVVM (Model-View-ViewModel) tüüpi raamistik. Lehe šablooni kujundamiseks kasutatakse HTML, 
mis siseldab Vue-spetsiifilist süntaksi, mille abil juhib raamistik lehe logikat. Vue rakendus koosneb SFC 
komponentidest, igas komponndis on eraldi defineritud lehe šabloon, skript ja stiil. Vue.js raamistik on
samuti laialt levinud ja selle kohta on võimalik Internetist piisavalt infot leida.

\textit{Angular} on MVC (Model-View-Controller) tüüpi raamistik. Angular-i projekt struktuurselt koosneb
moodulitest, komponentidest ja teenustest. Angular-is kasutatakse lehe šabloonides sarnaselt Vue raamistikule
HTML koodi Angular-spetsiifilise süntaksiga. 

Kuna üldiselt kõik raamistikud võimaldavad implementeerida kavandatavat funktsionaalsust, määravaks asjaoluks
on arendamisega tegeleva programmeerija eelistused. Kuigi töötamise kiirus on raamistikutel erinev, 
planeeritava rakenduse suurusjärgu kontekstis see faktor ei ole kriitiline. Toodud põhjendustel valitakse 
kasutajaliidese tehnoloogiaks React-i. Programmeerimiskeeleks peab valima TypeScript, mis on erinevalt
JavaScript-ist võimaldab teha tüübikirjeldust, tänu millele on programmi käitumine ettearvatavam, vigade
tõenäosus väiksem ja kood on üldiselt kvaliteetsem. 



\subsection{Serveriosa}
Serveril töötav \textit{backend} rakendus tegeleb kasutajaliidese päringute töötlusega ja andmete saatmisega.
Samuti \textit{backend} osa suhtleb andmebaasiga, küsides ja salvestades andmeid. Rakenduse serveriosa on võimalik
implementeerida kasutades järgmiseid programmeerimiskeeli:
\begin{itemize}
    \item \textit{PHP} -- väga popupaarne ja ka võrdlemisi lihtne programmeerimiskeel (avaldatud 1995), mis oli kohe alguselt välja mõeldud veebilehtede
genereerimiseks. Kuigi esialgu PHP kontseptsioon oli selline, et HTML-kood genereeriti serveril ja saadeti veebilehitsejale 
näitamiseks valmis kood (monoliitne arhitektuur), siis viimasel ajal kastutakse PHP ka REST-tüüpi veebirakendustes, kus serveril
töötav PHP programm saadab andmeid kasutajaliidese rakendusele JSON (või muul) kujul. Väga tugevaks eeliseks on see, et suur osa 
veebimajutust pakkuvaid teenuseid täna toetavad PHP keelt vaikimisi, mistõttu rakenduse paigaldamise protsess sellisel juhul on oluliselt lihtsam 
(koondub programmi failide kopeerimisele serverile). 
    \item \textit{Java} -- programmeerimiskeel (avaldatud 1995), mille arendamisega tegeleb Oracle, sobib suuremate REST-tüüpi veebirakenduste ehitamiseks,
kuid selle kasutusvaldkond on palju laiem kui ainult veebirakendused. Java on tugevalt ja staatiliselt tüübitud keel, 
mis on suureks eeliseks, kuna alandab vigade tekkimise tõenäosust, lisaks on see piisavalt kiire.  Samas eeldab see väga spetsiifilisi teadmisi programmeerialt ja ka rakenduse paigaldamine serverile on 
erinevalt PHP-st ka keerulisem, kuna projekti ehitamine eeldab palju lisategevusi. Java on kasutusel väga suure kasutajate hulgaga
infoüsteemides (sh. ka pangasüsteemid).
    \item \textit{C\#} -- programmeerimiskeel (avaldatud 2000), mille arendamisega tegeleb Microsoft ja mida nimetatakse ka Java analoogiks.
Keel on samuti tugevalt tüübitud ja ka programmi struktuur on Java-keelega analoogne. C\# samuti sobib suurte infosüsteemide
arendamiseks.
    \item \textit{Python} --  üldotstarbeline programmeerimiskeel, mille kasutusvaldkond on samuti väga lai kõigepealt sellele, et
keele süntaks on võrdlemisi lihtne ning vastavalt keel on lihtsam õpitav. Sobib rohkem väiksemate infosüsteemide jaoks, kuid teadaolevalt 
kasutatakse seda suuremates süsteemides. Keel on dünaamiliselt tüübitud, mida üldiselt saab pidada puuduseks, kuna võib põhjustada vigasid, aga ka 
pikaajalises perspektiivis vaadates on koodi loetavus kehvem.  
\end{itemize}

Veebirakenduse implementeerimiseks on otstarbekas kasutada analoogselt kasutajaliidesele raamistikku. Kõikidel ülaltoodud programmeerimiskeeltel 
eksisteerivad raamistikute lahendused, mis sobivad veebirakenduse serveriosa ehitamiseks.



\begin{itemize}
    \item \textbf{Laravel} - PHP keeles kirjutatud raamistik. \textit{lalala}
    \item \textbf{Spring} - Java keeles kirjutatud raamistik veebirakenduse ehitamiseks. Sisaldab palju erinevaid mooduleid 
(nt Spring Security - turvalisust tagav raamistiku osa, Spring MVC - MVC raamistik jm), mis tervikuna moodustavad 
väga tugevat infrastruktuuri suure infosüsteemi ehitamiseks.
    \item \textbf{.NET} - C\# keeles raamistik, mis samuti sobib REST veebirakenduste ehitamiseks. 
Vajalik funktsionaalsus on tagatav vastavate paketide paigaldamisega (nt EntityFrameworkCore - 
ORM raamistik, AspNetCore.Authentification.JwtBearer - JWT tokeni kaudu autentimise võimaldamine)
    \item \textbf{Django} - Python keeles kirjutatud raamistik. \textit{lalala}
\end{itemize}

Serveriosa tehnoloogiaks on valitud C\# keeles kirjutatud .NET raamistik, mitmel põhjusel: \textit{lalala}