\section{Tehnoloogiate valik}
Tehnoloogiate valimisel lähtutakse kaasaaegsetest veebirakendamise ehitamise pritsiibidest, arvestatakse
rakenduse loogika keerukust, võimalike kasutajate hulka, säilitavate andmete mahtusid ja infosüsteemi
edasise arengu perspektiive. 

Veebirakendusel peab olema selgelt eristatav serveriosa ja kasutajaliides. Vajadusel saab tulevikus 
implementeerida ka teised kasutajaliidesed, mis töötavad sama serveriosaga (näiteks: mobiilrakendus).
See tähendab, et infosüsteemi arhitektuur peab olema REST arhitektuurse stiili nõuete kohane. See eeldab
ka seda, et andmevahetus kasutajaliidese ja serveriosa vahel peab toimuma HTTP päringutega kasutades 
JSON (JavaScript Object Notation) formaati. JSON on JavaScript-i põhine andmevahetuse formaat, mis 
representeerib JavaScript-i andmeobjektid tekstilisel kujul, mida on võimalik saata HTTP 
päringutega üle veebi\cite{about_rest}.
Lisaks sellele on oluline, et kõik päringud on omavahel sõltumatud, see tähendab, et ühe 
päringu raames serveriosas alustatakse ja lõpetatakse kõik päringuga seotud protsessid, päringute 
vahel serveril puudub kliendiga seotud olek (\textit{stateless} protokoll)\cite{about_rest}.

Selleks, et süsteem saaks võimalikult pikemat aega töötama ilma tehnoloogiate uuenduse vajaduseta,
peab kasutusele võtma võimalikult uued, aga samas ka stabiilsed ja pikema toega lahenduste versioonid. 
 

\subsection{Kasutajaliides}
Kasutajaliides implementeeritakse üheleherakendusena (SPA). SPA tehnoloogia võimaldab minimeerida 
andmevahetuse mahtusid: serverilt küsitakse ja vastavalt kliendile saadetakse ainult need andmed, 
mida on hetkel tarvis. Arendatava infosüsteemi kontekstis see on oluline, sest kõiki arvutusi 
teostatakse serveril ning kliendile saadetakse andmed, mis on vajalikud tulemuste näitamiseks 
kasutajale. Iga uus tegevus kasutajaliideses, mis mõjutab tulemusi (uue kihi lisamine, 
kihtide järjekorra muutmine, arvutuse parameetrite muutmine jms.), tähendab uut päringut serverile. 
Samuti SPA tehnoloogia võimaldab muuta lehe sisu dünaamilisel viisil -- uuendatakse vaid lehe teatud
komponent ilma kogu lehekülje ümberlaadimise vajaduseta \cite{about_spa}. See on ka oluline, kuna tulemuste esitamist 
peab uuendama dünaamiliselt kohe peale arvutuse lähteandmete muutmist.

Üheleherakenduste implementeerimiseks kasutatakse JavaScript programmeerimiskeelt veebilehe dünaamilise 
loogika juhtimiseks. Soovitavalt kasutada JavaScript keele laiendust -- TypeScript, mis muudab JavaScript-i 
tugevalt ja staatiliselt tüübitud keeleks \cite{about_typescript}. Lehtede struktuuri ehitamiseks kasutatakse HTML (või raamistikust
sõltuv laiendatud HTML-i süntaks). Kujundust teostatakse CSS stiilireeglitega ja lihtsustamise mõttes
võetakse kasutusele ka vastavad teegid nt Bootstrap.

Kuigi on võimalik implementeerida loogikat kasutades puhtat JavaScript koodi, tänapäeval seda tehakse
harva. On olemas erinevad lahendused, mis oluliselt lihtsustavad rakenduse ehitamise protsessi, kuid 
nõuavad ka spetsiifilisi teadmisi. Raamistiku kasutuselevõtt olulisel määral vähendab koodi kirjutamist,
kuna raamistik ise haldab loogikat, mis on seotud näiteks \textit{Routing}-uga, turvalisusega, komponentide
genereerimise ja uuendamisega. Spetsiifilise funktsionaalsuse jaoks kasutatakse eraldi pluginaid ja teeke. Näiteks
päringute saatmise ja serveri vastuse töötluseks kasutatakse \textit{Axios} -- teek, mida saab kasutada 
erinevate raamistikutega.

Üheleherakenduse implementeerimiseks kõige sobilikud JavaScript raamistikud on \textit{React}, \textit{Vue.js}
ja \textit{Angular}. Kõikidel raamistikutel on oma eripärad alates projekti arhitektuurist kuni 
koodi süntaksini. 

\textit{React} on laialt levinud \textit{front-end} teek, mis kasutab JavaScript programmeerimiskeelt. 
Lehe šablooni kujundamiseks kasutatakse JSX (JavaScript XML). JSX on JavaScript-i laiendus, mis võimaldab
sisestada HTML koodi JavaScript-i programmi. Rakendus koosneb React-elementidest, mille oleku haldamisega teek
tegeleb ise. Elemendid on taaskasutatavad ning nendele antakse andmeid edasi andmeobjektide kujul (\textit{props}).
Kuna lahendus on populaarne -- selle kasutamise kohta on kogutud palju teavet ja kogemust veebis, mistõttu
probleemide tuvastamine ja lahenduste leidmine on piisavalt lihtne. Lisaks sellele eksisteerib palju pluginaid
ja teeke, mida saab React raamistikuga ühendada funktsionaalsuse laiendamiseks. 

\textit{Vue.js} on MVVM (Model-View-ViewModel) tüüpi raamistik. Lehe šablooni kujundamiseks kasutatakse HTML, 
mis siseldab Vue-spetsiifilist süntaksi, mille abil juhib raamistik lehe logikat. Vue rakendus koosneb SFC 
komponentidest, igas komponndis on eraldi defineritud lehe šabloon, skript ja stiil. Vue.js raamistik on
samuti laialt levinud ja selle kohta on võimalik Internetist piisavalt infot leida.

\textit{Angular} on MVC (Model-View-Controller) tüüpi raamistik. Angular-i projekt struktuurselt koosneb
moodulitest, komponentidest ja teenustest. Angular-is kasutatakse lehe šabloonides sarnaselt Vue raamistikule
HTML koodi Angular-spetsiifilise süntaksiga. 

Kõik ülaltoodud raamistikud toetavad ka \textit{TypeScript}-i kasutamist. Oma funktsionaalsuse seisukohalt
kõik toodud raamistikud võimaldavad implementeerida kavandatavat funktsionaalsust (\textit{routing}, 
oleku juhtimine, komponentide dünaamiline uuendamine), seega määravaks asjaoluks on arendamisega tegeleva 
programmeerija eelistused. Kuigi töötamise kiirus on raamistikutel erinev, planeeritava rakenduse suurusjärgu
kontekstis see faktor ei ole kriitiline. 

Toodud põhjendustel valitakse kasutajaliidese tehnoloogiaks React-i. Programmeerimiskeeleks peab valima TypeScript, 
mis on erinevalt JavaScript-ist võimaldab teha tüübikirjeldust, tänu millele on programmi käitumine ettearvatavam, 
vigade tõenäosus väiksem ja kood on üldiselt kvaliteetsem. React on populaarne lahendus, seetõttu eksisteerib palju
teeke, pluginaid ja leindusi, mida tõneäoliselt saab kasutusele võtta. Kasutada peab React viimane versioon,
mis on käesoleva töö koostamise hetkel v18.2.


\subsection{Serveriosa}
Serveril töötav \textit{backend} rakendus tegeleb kasutajaliidese päringute töötlusega ja andmete saatmisega.
Samuti \textit{backend} osa suhtleb andmebaasiga, küsides ja redigeerides andmeid. Rakenduse serveriosa on võimalik
implementeerida kasutades järgmiseid programmeerimiskeeli:
\begin{itemize}
    \item \textit{PHP} -- popupaarne ja ka võrdlemisi lihtne programmeerimiskeel (avaldatud 1995. aastal), 
    mille otstarve oli kohe alguselt suunatud veebilehtede ehitamiseks. Kuigi esialgu PHP kontseptsioon oli selline, 
    et HTML-kood genereeriti serveril ja saadeti veebilehitsejale iga kord uuesti näitamiseks (monoliitne arhitektuur), 
    siis viimasel ajal kastutakse PHP ka REST-tüüpi veebirakendustes, kus serveril töötav PHP programm saadab 
    andmeid kasutajaliidese rakendusele JSON (või muul) kujul. Tugevaks eeliseks on see, et suur osa veebimajutust 
    pakkuvaid teenuseid toetavad täna PHP keelt vaikimisi, mistõttu rakenduse paigaldamise protsess sellisel juhul 
    on oluliselt lihtsam (koondub programmi failide kopeerimisele serverile).
    \item \textit{Java} -- objektorienteeritud programmeerimiskeel (avaldatud 1995. aastal), mille arendamisega 
    tegeleb Oracle. Keel sobib suuremate REST-tüüpi veebirakenduste ehitamiseks, kuid selle kasutusvaldkond on palju laiem kui 
    ainult veebirakendused. Java on tugevalt ja staatiliselt tüübitud keel, mis on suureks eeliseks, kuna alandab
    vigade tekkimise tõenäosust, lisaks on see piisavalt kiire.  Samas eeldab see spetsiifilisi teadmisi 
    programmeerialt ja ka rakenduse paigaldamine serverile on erinevalt PHP-st ka keerulisem, kuna projekti 
    ehitamine eeldab palju lisategevusi. Java on kasutusel väga suure kasutajate hulgaga infoüsteemides (sh. ka pangasüsteemid).
    \item \textit{C\#} -- objektorienteeritud programmeerimiskeel (avaldatud 2000. aastal), mille arendamisega tegeleb Microsoft. 
    Keele süntaks ja programmi struktuuri põhimõtted on väga sarnased Java-le. Keel on tugevalt tüübitud ja 
    ka programmi struktuur on Java-keelega analoogne. C\# samuti sobib suurte infosüsteemide ehitamiseks.
    \item \textit{Python} --  üldotstarbeline programmeerimiskeel, mille kasutusvaldkond on lai -- programmeerimise 
    õpetamist koolilastele kuni suurte infosüsteemide ehitamiseni -- tänu kõigepealt sellele, et keele süntaks on 
    võrdlemisi lihtne ning vastavalt keel on kergemini õpitav. Keel on dünaamiliselt tüübitud, mida erinevates s
    ituatsioonides saab pidada nii eeliseks kui ka puuduseks.
\end{itemize}

Rakenduse ehitamiseks on otstarbekas kasutada analoogselt kasutajaliidesega raamistikku. Kõikidel ülaltoodud
programmeerimiskeeltel eksisteerivad raamistiku lahendused, mis sobivad veebirakenduse serveriosa ehitamiseks.

\begin{itemize}
    \item \textbf{Laravel} - PHP keeles kirjutatud raamistik, väga populaarne ja laialt levinud, funktsionaalsus 
    katab kõiki veebirakenduse ehitamise vajadusi.
    \item \textbf{Spring} - Java keeles kirjutatud raamistik veebirakenduse ehitamiseks. Sisaldab palju erinevaid 
    mooduleid (nt Spring Security - turvalisust tagav raamistiku osa, Spring MVC - MVC raamistik jm), mis 
    tervikuna moodustavad tugevat infrastruktuuri suurte infosüsteemi ehitamiseks.
    \item \textbf{.NET} - C\# keeles raamistik, mis samuti sobib REST veebirakenduste ehitamiseks. 
    Vajalik funktsionaalsus on tagatav vastavate paketide paigaldamisega (nt EntityFrameworkCore - 
    ORM raamistik, AspNetCore.Authentification.JwtBearer - JWT tokeni kaudu autentimise võimaldamine).
    Viimastel aastatel on raamistiku populaarsus oluliselt vähenenud.
    \item \textbf{Django} - Python keeles kirjutatud raamistik. On lihtne ja laia funktsionaalsusega,
    mis on kohe raamistikus saadaval ilma lisamoodulite paigaldamise vajaduseta.
\end{itemize}

Arendatava infosüsteemi seisukohalt peab tehnoloogia sobivuse hindama järgmiste aspektide seisukohalt: 
\begin{itemize}
    \item kiirus -- teenus peab piisavalt kiiresti teostama kõikvõimalikud arvutused, sh kasutades samal
    ajal andmeid andmebaasist.
    \item turvalisus -- raamistik peab (sisseehitud funktsionaalsus või laiendus) tegelema
    rakenduse turvalisusega sh kasutajate autentimisega. Raamistik peab toetama ka JWT tokeniga
    autentimist.
    \item ORM -- raamistikul peab olema \textit{Object Relational Mapping}-uga tegelev moodul, 
    selleks et lihtsustada andmebaasi andmetega tegutsemist koodis.
    \item arendaja oskused -- infosüsteemi arendamisega tegeleval ressurssil peavad olema piisavalt teadmisi
    ja kogemusi raamistikuga
\end{itemize} 

Raamistikute vastavus eeltoodud kriteeriumitele on toodud tabelis \ref{tab:requirements}.
\begin{longtable}{|p{3cm}|p{2.5cm}|p{2.5cm}|p{2.5cm}|p{2.5cm}|}
	\caption{\it{Backend raamistikute võrdlus}}
	\label{tab:requirements}\\ \hline
	\textbf{Raamistik} &  \textbf{Kiirus} & \textbf{Turvalisus}  & \textbf{ORM} & \textbf{Oskused} \\
	\hline
	\endhead
	\endfoot
	\hline
	\endlastfoot
Laravel & + & + & + & +/-  \\ \hline
.NET    & + & + & + & +  \\ \hline
Spring  & + & + & + & +/-  \\ \hline
Django  & + & + & + & -  \\ \hline
\end{longtable}

Kõik raamistikud omavad vajalikku funktsionaalsust arendatava infosüsteemi ehitamiseks, seetõttu valiku tegemist 
lähtutakse saadaval oleva programmeerimisressurssi oskuste tasemest erinevate raamistikutega. Sellest lähtuvalt
oli tehnoloogiaks valitud C\# keeles kirjutatud .NET raamistik.

\subsection{Andmebaasi juhtprogramm}
Kuna inosüsteemi äriloogika ei eelda suurte andmete mahtude säilimist, seetõttu ka andmebaasi juhtsüsteemi 
valiku osas on nõuded tagasihoidlikud: \textit{Open Source} tüüpi litsents, et välistada lisakulusid ning
võimalus ühendada andmebaasimootor valitud serveriosa raamistikuga (.NET). Kõige populaarsemad 
\textit{Open Source} litsentsiga andmebaasimootorid on:
\begin{itemize}
    \item MySQL
    \item PostgreSQL
    \item MariaDB
    \item MongoDB
    \item SQLite
\end{itemize}
Kõikidele ülaltoodud süsteemidele eksisteerivad juhtprogrammid .NET EF Core ühendamiseks, seetõttu võib kõik
toodud lahendused pidada sobilikuks. Valikul lähtutakse sellest, mis tehnoloogiaga on programmerijal rohkem teadmisi 
ja kogemusi. Antud juhul see on PostgreSQL. 

